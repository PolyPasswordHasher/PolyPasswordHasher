
%% bare_conf.tex
%% V1.3
%% 2007/01/11
%% by Michael Shell
%% See:
%% http://www.michaelshell.org/
%% for current contact information.
%%
%% This is a skeleton file demonstrating the use of IEEEtran.cls
%% (requires IEEEtran.cls version 1.7 or later) with an IEEE conference paper.
%%
%% Support sites:
%% http://www.michaelshell.org/tex/ieeetran/
%% http://www.ctan.org/tex-archive/macros/latex/contrib/IEEEtran/
%% and
%% http://www.ieee.org/

%%*************************************************************************
%% Legal Notice:
%% This code is offered as-is without any warranty either expressed or
%% implied; without even the implied warranty of MERCHANTABILITY or
%% FITNESS FOR A PARTICULAR PURPOSE! 
%% User assumes all risk.
%% In no event shall IEEE or any contributor to this code be liable for
%% any damages or losses, including, but not limited to, incidental,
%% consequential, or any other damages, resulting from the use or misuse
%% of any information contained here.
%%
%% All comments are the opinions of their respective authors and are not
%% necessarily endorsed by the IEEE.
%%
%% This work is distributed under the LaTeX Project Public License (LPPL)
%% ( http://www.latex-project.org/ ) version 1.3, and may be freely used,
%% distributed and modified. A copy of the LPPL, version 1.3, is included
%% in the base LaTeX documentation of all distributions of LaTeX released
%% 2003/12/01 or later.
%% Retain all contribution notices and credits.
%% ** Modified files should be clearly indicated as such, including  **
%% ** renaming them and changing author support contact information. **
%%
%% File list of work: IEEEtran.cls, IEEEtran_HOWTO.pdf, bare_adv.tex,
%%                    bare_conf.tex, bare_jrnl.tex, bare_jrnl_compsoc.tex
%%*************************************************************************

% *** Authors should verify (and, if needed, correct) their LaTeX system  ***
% *** with the testflow diagnostic prior to trusting their LaTeX platform ***
% *** with production work. IEEE's font choices can trigger bugs that do  ***
% *** not appear when using other class files.                            ***
% The testflow support page is at:
% http://www.michaelshell.org/tex/testflow/



% Note that the a4paper option is mainly intended so that authors in
% countries using A4 can easily print to A4 and see how their papers will
% look in print - the typesetting of the document will not typically be
% affected with changes in paper size (but the bottom and side margins will).
% Use the testflow package mentioned above to verify correct handling of
% both paper sizes by the user's LaTeX system.
%
% Also note that the "draftcls" or "draftclsnofoot", not "draft", option
% should be used if it is desired that the figures are to be displayed in
% draft mode.
%
\documentclass[conference]{IEEEtran}
% Add the compsoc option for Computer Society conferences.
%
% If IEEEtran.cls has not been installed into the LaTeX system files,
% manually specify the path to it like:
% \documentclass[conference]{../sty/IEEEtran}





% Some very useful LaTeX packages include:
% (uncomment the ones you want to load)

%\usepackage{epsfig,listings, algorithm, algorithmic, graphicx, caption, cite}
\usepackage{algorithmic, graphicx, caption, cite}

%\DeclareCaptionType{copyrightbox}
\usepackage[table]{xcolor}
%\usepackage{multirow}
%\usepackage{booktabs}
\usepackage{url}


% *** MISC UTILITY PACKAGES ***
%
%\usepackage{ifpdf}
% Heiko Oberdiek's ifpdf.sty is very useful if you need conditional
% compilation based on whether the output is pdf or dvi.
% usage:
% \ifpdf
%   % pdf code
% \else
%   % dvi code
% \fi
% The latest version of ifpdf.sty can be obtained from:
% http://www.ctan.org/tex-archive/macros/latex/contrib/oberdiek/
% Also, note that IEEEtran.cls V1.7 and later provides a builtin
% \ifCLASSINFOpdf conditional that works the same way.
% When switching from latex to pdflatex and vice-versa, the compiler may
% have to be run twice to clear warning/error messages.






% *** CITATION PACKAGES ***
%
%\usepackage{cite}
% cite.sty was written by Donald Arseneau
% V1.6 and later of IEEEtran pre-defines the format of the cite.sty package
% \cite{} output to follow that of IEEE. Loading the cite package will
% result in citation numbers being automatically sorted and properly
% "compressed/ranged". e.g., [1], [9], [2], [7], [5], [6] without using
% cite.sty will become [1], [2], [5]--[7], [9] using cite.sty. cite.sty's
% \cite will automatically add leading space, if needed. Use cite.sty's
% noadjust option (cite.sty V3.8 and later) if you want to turn this off.
% cite.sty is already installed on most LaTeX systems. Be sure and use
% version 4.0 (2003-05-27) and later if using hyperref.sty. cite.sty does
% not currently provide for hyperlinked citations.
% The latest version can be obtained at:
% http://www.ctan.org/tex-archive/macros/latex/contrib/cite/
% The documentation is contained in the cite.sty file itself.






% *** GRAPHICS RELATED PACKAGES ***
%
\ifCLASSINFOpdf
  % \usepackage[pdftex]{graphicx}
  % declare the path(s) where your graphic files are
  % \graphicspath{{../pdf/}{../jpeg/}}
  % and their extensions so you won't have to specify these with
  % every instance of \includegraphics
  % \DeclareGraphicsExtensions{.pdf,.jpeg,.png}
\else
  % or other class option (dvipsone, dvipdf, if not using dvips). graphicx
  % will default to the driver specified in the system graphics.cfg if no
  % driver is specified.
  % \usepackage[dvips]{graphicx}
  % declare the path(s) where your graphic files are
  % \graphicspath{{../eps/}}
  % and their extensions so you won't have to specify these with
  % every instance of \includegraphics
  % \DeclareGraphicsExtensions{.eps}
\fi
% graphicx was written by David Carlisle and Sebastian Rahtz. It is
% required if you want graphics, photos, etc. graphicx.sty is already
% installed on most LaTeX systems. The latest version and documentation can
% be obtained at: 
% http://www.ctan.org/tex-archive/macros/latex/required/graphics/
% Another good source of documentation is "Using Imported Graphics in
% LaTeX2e" by Keith Reckdahl which can be found as epslatex.ps or
% epslatex.pdf at: http://www.ctan.org/tex-archive/info/
%
% latex, and pdflatex in dvi mode, support graphics in encapsulated
% postscript (.eps) format. pdflatex in pdf mode supports graphics
% in .pdf, .jpeg, .png and .mps (metapost) formats. Users should ensure
% that all non-photo figures use a vector format (.eps, .pdf, .mps) and
% not a bitmapped formats (.jpeg, .png). IEEE frowns on bitmapped formats
% which can result in "jaggedy"/blurry rendering of lines and letters as
% well as large increases in file sizes.
%
% You can find documentation about the pdfTeX application at:
% http://www.tug.org/applications/pdftex





% *** MATH PACKAGES ***
%
%\usepackage[cmex10]{amsmath}
% A popular package from the American Mathematical Society that provides
% many useful and powerful commands for dealing with mathematics. If using
% it, be sure to load this package with the cmex10 option to ensure that
% only type 1 fonts will utilized at all point sizes. Without this option,
% it is possible that some math symbols, particularly those within
% footnotes, will be rendered in bitmap form which will result in a
% document that can not be IEEE Xplore compliant!
%
% Also, note that the amsmath package sets \interdisplaylinepenalty to 10000
% thus preventing page breaks from occurring within multiline equations. Use:
%\interdisplaylinepenalty=2500
% after loading amsmath to restore such page breaks as IEEEtran.cls normally
% does. amsmath.sty is already installed on most LaTeX systems. The latest
% version and documentation can be obtained at:
% http://www.ctan.org/tex-archive/macros/latex/required/amslatex/math/





% *** SPECIALIZED LIST PACKAGES ***
%
%\usepackage{algorithmic}
% algorithmic.sty was written by Peter Williams and Rogerio Brito.
% This package provides an algorithmic environment fo describing algorithms.
% You can use the algorithmic environment in-text or within a figure
% environment to provide for a floating algorithm. Do NOT use the algorithm
% floating environment provided by algorithm.sty (by the same authors) or
% algorithm2e.sty (by Christophe Fiorio) as IEEE does not use dedicated
% algorithm float types and packages that provide these will not provide
% correct IEEE style captions. The latest version and documentation of
% algorithmic.sty can be obtained at:
% http://www.ctan.org/tex-archive/macros/latex/contrib/algorithms/
% There is also a support site at:
% http://algorithms.berlios.de/index.html
% Also of interest may be the (relatively newer and more customizable)
% algorithmicx.sty package by Szasz Janos:
% http://www.ctan.org/tex-archive/macros/latex/contrib/algorithmicx/




% *** ALIGNMENT PACKAGES ***
%
%\usepackage{array}
% Frank Mittelbach's and David Carlisle's array.sty patches and improves
% the standard LaTeX2e array and tabular environments to provide better
% appearance and additional user controls. As the default LaTeX2e table
% generation code is lacking to the point of almost being broken with
% respect to the quality of the end results, all users are strongly
% advised to use an enhanced (at the very least that provided by array.sty)
% set of table tools. array.sty is already installed on most systems. The
% latest version and documentation can be obtained at:
% http://www.ctan.org/tex-archive/macros/latex/required/tools/


%\usepackage{mdwmath}
%\usepackage{mdwtab}
% Also highly recommended is Mark Wooding's extremely powerful MDW tools,
% especially mdwmath.sty and mdwtab.sty which are used to format equations
% and tables, respectively. The MDWtools set is already installed on most
% LaTeX systems. The lastest version and documentation is available at:
% http://www.ctan.org/tex-archive/macros/latex/contrib/mdwtools/


% IEEEtran contains the IEEEeqnarray family of commands that can be used to
% generate multiline equations as well as matrices, tables, etc., of high
% quality.


%\usepackage{eqparbox}
% Also of notable interest is Scott Pakin's eqparbox package for creating
% (automatically sized) equal width boxes - aka "natural width parboxes".
% Available at:
% http://www.ctan.org/tex-archive/macros/latex/contrib/eqparbox/





% *** SUBFIGURE PACKAGES ***
%\usepackage[tight,footnotesize]{subfigure}
% subfigure.sty was written by Steven Douglas Cochran. This package makes it
% easy to put subfigures in your figures. e.g., "Figure 1a and 1b". For IEEE
% work, it is a good idea to load it with the tight package option to reduce
% the amount of white space around the subfigures. subfigure.sty is already
% installed on most LaTeX systems. The latest version and documentation can
% be obtained at:
% http://www.ctan.org/tex-archive/obsolete/macros/latex/contrib/subfigure/
% subfigure.sty has been superceeded by subfig.sty.



%\usepackage[caption=false]{caption}
%\usepackage[font=footnotesize]{subfig}
% subfig.sty, also written by Steven Douglas Cochran, is the modern
% replacement for subfigure.sty. However, subfig.sty requires and
% automatically loads Axel Sommerfeldt's caption.sty which will override
% IEEEtran.cls handling of captions and this will result in nonIEEE style
% figure/table captions. To prevent this problem, be sure and preload
% caption.sty with its "caption=false" package option. This is will preserve
% IEEEtran.cls handing of captions. Version 1.3 (2005/06/28) and later 
% (recommended due to many improvements over 1.2) of subfig.sty supports
% the caption=false option directly:
%\usepackage[caption=false,font=footnotesize]{subfig}
%
% The latest version and documentation can be obtained at:
% http://www.ctan.org/tex-archive/macros/latex/contrib/subfig/
% The latest version and documentation of caption.sty can be obtained at:
% http://www.ctan.org/tex-archive/macros/latex/contrib/caption/




% *** FLOAT PACKAGES ***
%
%\usepackage{fixltx2e}
% fixltx2e, the successor to the earlier fix2col.sty, was written by
% Frank Mittelbach and David Carlisle. This package corrects a few problems
% in the LaTeX2e kernel, the most notable of which is that in current
% LaTeX2e releases, the ordering of single and double column floats is not
% guaranteed to be preserved. Thus, an unpatched LaTeX2e can allow a
% single column figure to be placed prior to an earlier double column
% figure. The latest version and documentation can be found at:
% http://www.ctan.org/tex-archive/macros/latex/base/



%\usepackage{stfloats}
% stfloats.sty was written by Sigitas Tolusis. This package gives LaTeX2e
% the ability to do double column floats at the bottom of the page as well
% as the top. (e.g., "\begin{figure*}[!b]" is not normally possible in
% LaTeX2e). It also provides a command:
%\fnbelowfloat
% to enable the placement of footnotes below bottom floats (the standard
% LaTeX2e kernel puts them above bottom floats). This is an invasive package
% which rewrites many portions of the LaTeX2e float routines. It may not work
% with other packages that modify the LaTeX2e float routines. The latest
% version and documentation can be obtained at:
% http://www.ctan.org/tex-archive/macros/latex/contrib/sttools/
% Documentation is contained in the stfloats.sty comments as well as in the
% presfull.pdf file. Do not use the stfloats baselinefloat ability as IEEE
% does not allow \baselineskip to stretch. Authors submitting work to the
% IEEE should note that IEEE rarely uses double column equations and
% that authors should try to avoid such use. Do not be tempted to use the
% cuted.sty or midfloat.sty packages (also by Sigitas Tolusis) as IEEE does
% not format its papers in such ways.





% *** PDF, URL AND HYPERLINK PACKAGES ***
%
%\usepackage{url}
% url.sty was written by Donald Arseneau. It provides better support for
% handling and breaking URLs. url.sty is already installed on most LaTeX
% systems. The latest version can be obtained at:
% http://www.ctan.org/tex-archive/macros/latex/contrib/misc/
% Read the url.sty source comments for usage information. Basically,
% \url{my_url_here}.





% *** Do not adjust lengths that control margins, column widths, etc. ***
% *** Do not use packages that alter fonts (such as pslatex).         ***
% There should be no need to do such things with IEEEtran.cls V1.6 and later.
% (Unless specifically asked to do so by the journal or conference you plan
% to submit to, of course. )


% correct bad hyphenation here
\hyphenation{op-tical net-works semi-conduc-tor}


\begin{document}

%JAC: \widowpenalty=10000
%JAC: \clubpenalty=10000

\newcommand{\cappos}[1]{{\color{red} [JustinC: #1]}}
\newcommand{\eat}[1]{}

% enable this to de-anonymize!
%\newcommand{\showurlx}{{\url{https://polypasswordhasher.poly.edu}}}
\newcommand{\showurlx}{[redacted]}
%
% paper title
% can use linebreaks \\ within to get better formatting as desired
\title{PolyPasswordHasher: Protecting Passwords In The Event Of A Password File 
Disclosure}


% author names and affiliations
% use a multiple column layout for up to three different
% affiliations
%\author{\IEEEauthorblockN{Justin Cappos}
%\IEEEauthorblockA{Computer Science and Engineering\\
%NYU Poly\\
%Brooklyn, New York 11201\\
%Email: jcappos@poly.edu}}

\author{Author names removed for anonymous submission}

% conference papers do not typically use \thanks and this command
% is locked out in conference mode. If really needed, such as for
% the acknowledgment of grants, issue a \IEEEoverridecommandlockouts
% after \documentclass






% use for special paper notices
%\IEEEspecialpapernotice{(Invited Paper)}




% make the title area
\maketitle


%\subsection*{Abstract}
\begin{abstract}
\begin{abstract} 

Over the years, we have witnessed various password-hash database breaches that 
have affected small and large companies, with a diversity of users and budgets.
The industry standard, salted hashing (and even key stretching), has proven
to be insufficient protection against attackers who now have access to clusters of
GPU-powered password crackers.  Although there are various proposals for better securing
password storage, most do not offer the same adoption model 
(software-only, server-side) as salted hashing, which may impede adoption.

In this paper, we present \PPH, a software-only, server-side password storage
mechanism that requires minimal additional work for the server, but 
exponentially increases the attacker's effort.  \PPH uses a threshold 
cryptosystem to interrelate stored password data so that passwords cannot be 
individually cracked.  Our analysis shows that \PPH is memory and storage
efficient, hard to crack, and easy to implement.  In many realistic scenarios, 
cracking a \PPH-enabled database would be infeasible even for an adversary
with millions of computers.

\end{abstract}

\begin{IEEEkeywords}
    password, password-hashing, authentication, password cracking, cryptography.
\end{IEEEkeywords}

\end{abstract}
%\begin{abstract}
%%\boldmath
%The abstract goes here.
%\end{abstract}
% IEEEtran.cls defaults to using nonbold math in the Abstract.
% This preserves the distinction between vectors and scalars. However,
% if the conference you are submitting to favors bold math in the abstract,
% then you can use LaTeX's standard command \boldmath at the very start
% of the abstract to achieve this. Many IEEE journals/conferences frown on
% math in the abstract anyway.

% no keywords




% For peer review papers, you can put extra information on the cover
% page as needed:
% \ifCLASSOPTIONpeerreview
% \begin{center} \bfseries EDICS Category: 3-BBND \end{center}
% \fi
%
% For peerreview papers, this IEEEtran command inserts a page break and
% creates the second title. It will be ignored for other modes.
\IEEEpeerreviewmaketitle

\section{Introduction}
\label{SEC:introduction}

Password database thefts are increasingly common and continue to put users at
risk. Both individuals and businesses want passwords to be stored efficiently,
yet in a way that minimizes the impact of a password database disclosure. Major
corporations, in particular, have encountered significant challenges in trying
to protect their customers' and clients' identities and accounts. Reflecting
the reach of the problem, a hacker on a Russian forum was selling a database
with more than a billion password hashes obtained from different password
databases, primarily through SQL injection \cite{miranteTR13}. Some companies
and organizations that have been successfully hacked include: the Social
Security Administration, Hotmail, LastFM, Formspring, ScribD, the New York
Times, Nvidia, Evernote, LinkedIn, Billabong, Gawker, Linode, ABC, Yahoo!,
eHarmony, LivingSocial, and Twitter~\cite{miranteTR13}. Globally -- despite
ongoing efforts to enhance password security and adhere to best practices --
password database breaches continue. 

Notwithstanding new technologies such as authentication tags, biometric
technologies, and NFC-capable devices, the most prevalent user authentication
method remains the user password. Security experts have long advocated that
user passwords not be stored in plain text but rather, that they be reduced to
salted hashes before storing.  A salted hash consists of a secure hash and a
salt.  The secure hash acts as a one-way function that ensures that an attacker
cannot easily read the plain-text passwords from disk. Salting inserts a random
value that complicates the use of tables that allow hackers to immediately look
up passwords to match most hashes.  Storing passwords with a salted hash is
widely considered to be the best practice because of the level of protection
offered and the ease of implementation (e.g., salted hashing requires no
additional hardware or client software).

However, storing passwords with a salted hash is not a panacea.  That is, when
stored password data has been compromised, attackers have proven themselves adept at
quickly cracking large numbers of passwords, even when salted-password hashes
comprise the stored data. For example, Troy Hunt, a security researcher, showed
that cracking 60\% of a 40,000-entry salted-SHA1 database can be done in a
couple of hours, using a couple of GPU’s and an average computer~\cite{thunt-hashing}. 

This paper presents PolyPasswordHasher, a new technique that makes cracking
individual passwords infeasible because the stored password-entries are 
interdependent.  We leverage cryptographic hashing and threshold cryptography 
to combine 
password hash data with shares so that users unknowingly protect each other's password
data.  With a small amount of additional work by the server, 
PolyPasswordHasher increases, by many orders of magnitude, the time an
attacker needs to crack passwords.  

PolyPasswordHasher is designed to be easy for organizations to adopt.  Much
like salted hashing, PolyPasswordHasher is a software-only, single-server
enhancement that can be deployed on a server without any changes to clients.
PolyPasswordHasher relies on simple primitives that are efficient from a
storage, memory, and computational standpoint.  It integrates with other forms
of authentication including: OAuth, hardware tokens, two factor authentication,
and fingerprint authentication. PolyPasswordHasher is also easy to implement;
two outside developers, whom we had no prior contact with, independently
built implementations of PolyPasswordHasher, in different programming
languages.  Deploying PolyPasswordHasher is a straightforward software
installation on a server.

The contributions of this paper are as follows: 

\begin{itemize}
    \item We describe our design of PolyPasswordHasher, a server-side, 
software-only scheme that protects against an attacker who is able to 
read all persistent storage on a server, including the complete password file.
    \item We demonstrate the \PPH has similar performance, storage, and
memory requirements to deployed systems.
    \item  We describe a mechanism to shield some user accounts, who
may be untrustworthy or choose poor passwords, as long as the passwords 
from a set of \thresholdaccounts are not cracked.
    \item We analyze different practical scenarios and analyze the properties 
and limitations of \PPH in these environments.
\end{itemize}

This paper is organized into eight sections. Sections~\ref{SEC:threat-model}
and~\ref{SEC:background} introduce the threat model and the theoretical
background of password storage mechanisms.  Sections~\ref{SEC:design}
and~\ref{SEC:implementation} provide the design specification and 
implementation of \PPH. 
%Section~\ref{SUBSEC:deployment} provides information about our deployed django
%application and it’s specific design considerations to provide a feel of how an
%actual PolyPasswordHasher implementation works.  
Section~\ref{SEC:evaluation} evaluates \PPH's suitability and resilience
to cracking in a variety of scenarios; it also
provides information about the implementation’s performance for different kinds
of operations. 
We describe the relation to \PPH with other related work in 
Section~\ref{SEC:related-work}.
Finally, Section~\ref{SEC:conclusion} summarizes and anaylzes our findings 
described in the paper.



\section{Threat Model}
\label{sec-threatmodel}

Suppose that a server wants to validate passwords for user accounts.   
%registering an account, the user and server will agree on a salted 
%will register an account  provide the server with a salted hash of 
%the password.   
%Note that it is irrelevant which party generates the sal(In practice this is more commonly done on the server side, but 
%the salt generation and hash can be performed by either party.)
We assume that:

\begin{itemize}

\item An attacker can read all data that is persisted on disk including
the password file.

\item The server can fail and need to be restarted by administrators.   All
state that is kept in memory is lost at this point.   The server must
restart using only the (attacker visible) data from disk.   (While the basic
technique needs a threshold of correct passwords to perform verification, 
Section~\ref{sec-partial} discusses an extension to immediately verify
passwords upon restart.)

\item The attacker can have a priori knowledge of the correct
passwords for some number of user accounts.   We assume
that the administrator has configured the threshold to be larger than this
value.   (Note that a technique to have protected user accounts that
do not count toward the threshold is described in 
Section~\ref{sec-thresholdless}.)

\item 
The attacker {\bf cannot} read arbitrary memory for all processes on the 
server.  
If an attacker can read arbitrary memory, then the attacker can observe
the plain-text passwords as they are entered and so any form of secure password
storage can be bypassed.

The attacker cannot read arbitrary memory in many types of reported attacks
including compromises of password file data stored on 
backup media, web server misconfiguration, or information obtained through SQL 
injection attacks --- the primary vector for password file 
compromises~\cite{passwordresearchblog, miranteTR13}.

\item
Even if these guarantees are not met, the goal is to still retain
strong security.   Regardless of the attacker's knowledge, the protections
provided in this work must never be worse than the best practices used today 
(salted hashes of passwords).

\end{itemize}

Limitations and strengths of alternative solutions that take this threat 
model into account are discussed in Section~\ref{sec-expecteduse}.


\section{Background}
\label{SEC:background}

We briefly review the relevant properties of cryptographic $(k, n)$-threshold
schemes.  Although the specific $(k,n)$-threshold scheme that is used 
is not fundamental to our
work, we describe Shamir Secret Sharing \cite{shamir1979share}, which we use to develop
explicit examples within the text. 

\emph{The $(k, n)$-threshold scheme}

Threshold schemes protect secret information (usually a key) by deriving $n$
different shares from this information.  A threshold scheme describes how any $k$
shares (from a set of n total shares) can be used to recover an original
secret.  The number of needed shares, $k$, is called the threshold.  If fewer than
$k$ shares are known, no information about the secret is provided.

Shamir Secret Sharing is an algorithm that describes how a secret is divided
into a set of $n$ shares.  If a threshold $k$ of shares are input (specified when the 
secret is divided), the original secret can be reconstructed. To hide
a secret, Shamir Secret Sharing computes $k - 1$ random coefficients for a $k - 1$
degree polynomial $f(x)$ in a finite field (commonly GF-256 or GF-65536). The $k$th
term (commonly the constant term) contains the secret. To compute a share, a
value between 1 and the order of the field is chosen. The polynomial is
evaluated with $x$ equal to the share value, where the terms $x$ and $f(x)$ are used
as the share. To reconstruct the secret from at least $k$ shares, a party can
interpolate the values in the finite field to find the constant term (i.e., the
secret). In practice, interpolation is often computationally optimized so that
only the constant term is recovered.

Suppose that a secret, 235, is to be hidden so that it can only be
reconstructed if three shares are provided. Because the threshold is three, two
random terms are first generated (24 and 182) to build a GF-256 polynomial,
such as $f(x) = 24x^2 + 182x + 235$. Shares can then be generated by computing $x$
and $f(x)$, such as: (1, 92), (2, 148), (3, 37), (4, 69), etc. A party that has
at least three shares can interpolate to reconstruct the full polynomial of
$f(x)$ and thus, the secret (235). 

It is possible to generate additional shares after recovering the secret, (e.g.,
if Lagrange interpolation is performed during reconstruction). This makes share
recovery slightly more computationally complex but also makes it possible (and
efficient) to generate additional shares simply by evaluating f(x) for the
specified share.  This means that all shares do not need to be created
initially --- they may instead be added (or recovered) after the secret has
been reconstructed.

In many cases, the secret will be larger than the size of the finite field.  A
large secret can be stored by breaking it into segments that are the size of
the finite field (often one or two bytes) and applying the above technique
separately to each segment.  The same share number, $x$, is typically used for
each share $f_i(x)$.  This simplifies --- and effectively hides --- the fact that a
secret has only a limited size.  An integrity check can be added to detect whether
an incorrect share has been provided.  As was previously described, when given a set
of any k distinct shares, whether valid or invalid, Shamir Secret Sharing will
produce a polynomial of the appropriate length. This means that if any share is
invalid, its resulting polynomial will be incorrect. To avoid this problem,
implementations of Shamir Secret Sharing typically store an integrity check by
appending the hash of the secret to the secret; this check detects if an
incorrect share has been provided.  When the shares are reconstructed, the
additional integrity check provides verification that the correct shares were
given.



\section{PolyHashing: Integrity Verification}
\label{sec-polyhashing}

\begin{figure}
\center
% We need pdflatex with png images. Plain latex would fail here.
% So, we conditionally include this so we can still build with latex,
% just without this image.
%\begin{minipage}[b]{\linewidth}
\hspace{-4mm}
\includegraphics[width=1.05\columnwidth]{figures/polyhashing.png}
%\end{minipage}
\caption{This figure demonstrates inserting data and integrity verification 
for PolyHashing.   In the case of data verification, the caller performs 
PolyHashing verification on values simultaneously.  Only if at least the 
threshold of correct values are provided (3), can 
the threshold cryptosystem be decoded and any value be checked. Note that the
caller does not know whether data values are correct or invalid when calling
the verification routine.   Only the data in the dashed box is persisted
on disk.}
\label{fig-polyhashing}
\end{figure}


%Traditionally, a threshold cryptosystem is used to hide 
%information so that multiple parties must come together to recover the 
%original data.  With PolyHashing, there is a single party with all data, but
%the threshold system obscures hash information so that values may only be 
%checked if a threshold of correct values are known.
%To achieve these properties, PolyHashing combines two techniques: a threshold
%cryptosystem and a hash function.   


PolyHashing is a general technique for concurrently verifying the integrity of 
a set of values.% (the next section discusses how to apply PolyHashing to the
%specific problem of protecting passwords).   
The purpose of PolyHashing is 
much like using a traditional hashing function and storing 
the hash of a value $h(v_i)$ in the store at a specific index $i$.   However, 
with PolyHashing, the hash values are blinded so that
integrity verification may only be performed if a threshold $k$ of the values 
are provided.   If less than the threshold of correct $h(v)$ items 
are provided, {\bf no information about which values are correct is leaked.}   


These properties are provided by leveraging a threshold cryptosystem 
(Figure~\ref{fig-polyhashing}).   Each entry in a PolyHashing store is XORed 
with a share.   If a threshold of correct values are provided, then the 
attacker can recover enough shares to recover the secret.   By using
interpolation, all shares to be computed and thus any hash can be 
validated.  As such, a PolyHashing store has information-theoretic 
confidentiality when 
less than the threshold of values are known, but can validate the 
integrity of values individually if at least $k$ of them are known.


When a blank PolyHashing store is created, the threshold $k$ is given.  
A cryptographically random value is used to seed all shares in the
threshold cryptosystem.  For example, when using Shamir Secret Sharing for 
PolyHashing, the constant term (which is typically the stored secret) is 
randomly generated.   In this case, the purpose of the threshold cryptosystem 
is not to store a secret, but instead to act like a one-time pad to obscure 
the hashes that are stored.   As such the length of shares generated by 
the threshold cryptosystem should be equal to the length of the hashes.


To add a secret to the store (the top of Figure~\ref{fig-polyhashing}), one 
will compute an index $i$ (to 
later locate which secret in the store corresponds to this entry) and
the hash of the data.   The share value can be stored in the table, or
alternatively $i$ can also be used as the 
share number in the Shamir Secret Store for computing the $f(x)$ values of the
shares.   To store an item, the party will add an entry that includes the 
identifier, share number, and the XOR of the $f(x)$ values 
with the hash.   %If desired, a single secret can have more
%than one share by choosing a unique  and share number and repeating
%the process.   



If a party possesses the PolyHashing store, they cannot validate any hash 
value without at least $k$ correct hash values (the bottom left of 
Figure~\ref{fig-polyhashing}) because they do not have enough hashes
to recover the threshold
cryptosystem's shares.  When a party has $k$ or more hash 
values (bottom right of Figure~\ref{fig-polyhashing}), they can 
uncover all shares of the threshold cryptosystem and validate all hash
values.   As a result, other requests can be validated by computing
the hash of the value $h(v_i)$, XORing the $i$th share, and
comparing the result with the stored data in the PolyHashing store.

Note that the threshold cryptosystem's shares are not stored on disk or 
otherwise transmitted.   Once enough correct values are known by a party, by 
XORing the $h(v)$ values with the data in the PolyHashing store, one can 
recover a threshold of shares.   This allows the party to reconstruct all
data in the threshold cryptosystem's store.   By keeping this information
only in memory on a running system, an attacker that can read the disk 
contents but does not know a threshold of correct values cannot validate
the correctness of values individually.

More formally, given an information-theoretic $(k,n)$-threshold cryptosystem 
with shares $S=\{s_0, ... s_n\}$ that stores values $H=\{h(d_0), ... h(d_n)\}$.
Suppose that $length(s_i) = length(h(x))$. A PolyHashing 
store $Z$ has items $Z=\{z_0, ...z_n\}$ such that $z_i= s_i \oplus h(d_i)$.   

{\bf Theorem:} $Z$ provides information-theoretic 
privacy for unknown items in $H$ when $k-1$ items from $H$ are known.


{\bf Proof:}
Suppose that $h(d_j)$ is an item that is not in the set of $k-1$ known items.
Since $S$ is a $(k,n)$-threshold cryptosystem and the observer knows only 
$k-1$ items, they only know $k-1$ shares of $S$.   Thus they cannot recover 
$S$ from the known shares of $S$.   Therefore $s_j$ has information-theoretic 
privacy.   Since $z_j$ is 
the result of XORing $s_j$ with another value of the same length, $s_j$ acts 
like a one-time pad for $h(d_j)$.   Thus, $z_j$ reveals no information
about $h(d_j)$.   However, since the choice of $j$ was 
arbitrary, the same argument applies to all of the $n-(k-1)$ unknown items in 
$H$ and clearly no information about $h(d_j)$ is revealed by any $z_i$ where
$i \neq j$.  
$\blacksquare$





\input{polypasswordhasher}

\subsection{PolyPassHash Extensions}
\label{sec-extensions}

There are three major limitations of PolyPassHash as described in the previous
section.   
First, for some systems like Facebook or gmail, an attacker may control a
huge number of accounts.  In practice there are really two 
types of accounts: those that should count toward the threshold (likely 
administrators or power users) and those that should not.   This section
describes an extension to support ``thresholdless'' accounts that do not 
count toward the threshold.
Second, there is not a way to handle non-password authentication schemes
like biometrics, private key authentication, etc.   This section 
discusses how to integrate into existing mechanisms.   
Third, the system
must have a threshold of correct passwords before any can be authenticated.
This may cause logins to be delayed for an unacceptable time after a restart.
This section discusses an extension that leaks partial information about salted 
password hashes, but allows immediate authentication upon restart.   As a 
result, PolyPassHash still provides an exponential security increase, but 
no longer must wait for account logins upon restart.




\subsubsection{Thresholdless Accounts}
\label{sec-thresholdless}

A desirable property is the ability to handle user accounts that should
not count toward the threshold.  For example, gmail and Facebook have 
a huge number of users that are untrusted and can create their accounts 
automatically.   None of these user accounts, no matter how many, should be
allowed to count toward the threshold for verifying other accounts.   

We will support accounts that should not count toward the threshold by
computing their salted hash and then encrypting this data with
a symmetric cryptographic cipher (AES).   If the server knows the AES key, 
it can clearly authenticate thresholdless users.
However, if the AES key is stored on
disk, an attacker that compromises the password file could compromise the
AES key too.   

To protect the AES key while stored on disk we will use a slightly modified
version of the PolyHashing store algorithm.   This store is identical to 
the main algorithm except
that the lowest order term (the constant term) of the polynomial represents 
a portion of the encryption key, as is shown in the bottom half of 
Figure~\ref{fig-polypasshashoverview}.   (The key is stored in the same manner
as traditional Shamir Secret Sharing
where the constant term of the Polynomial encodes the secret.)
A party that knows the threshold of administrator passwords can XOR those
values to recover sufficient shares to reconstruct the polynomial (and thus
the key).  
By the properties of Shamir Secret Sharing, the key has information-theoretic 
privacy from an attacker unless they possess a threshold of account passwords.



Creation of a thresholdless account involves a similar series of steps to
other PolyPassHash passwords.   As before the system obtains the 
salted hash for the password.   However, the salted cryptographic hash of 
the password is encrypted with AES instead of XORing it with a share.
To indicate this is a thresholdless account, the share field can 
be set to 0 to indicate that this is not a normal PolyPassHash 
store.   (Note that the share 0 will not be used normally in Shamir secret
sharing.)   When the user attempts to login, if a large enough 
threshold of passwords has been entered, the AES key is available to decrypt 
the hash, allowing the user to  be authenticated.   However,
if the stored threshold data is disclosed, it is not useful to an attacker
unless they can recover the AES key by knowing a threshold of
administrator passwords.


If desired by an administrator, accounts can be switched between thresholdless
and threshold without user intervention.
In both the case of threshold and thresholdless accounts, a server with
a threshold of shares knows the information necessary to recover the salted 
secure hash.   This salted secure hash can then be re-encoded in the 
alternative manner and the new entry can be stored.

\subsubsection{Handling Alternative Authentication Mechanisms}

Passwords are not the only mechanism for logging into modern systems.   For
example, it is common for users to use a private key to login over {\tt
ssh}, biometric-based authentication holds substantial promise, 
smart cards are often used to hide user credentials, and single
sign-in systems like OAuth and OpenID are commonplace and provided by 
many major websites.   Any practical protection mechanism needs to operate
in conjunction with such techniques.

Handling non-password login for \emph{thresholdless} user accounts is trivial 
because
it requires no changes to the system.   Non-password login can be handled 
using the existing login mechanisms without any modification.
This is especially important for techniques like OAuth and OpenID which 
are almost certainly not going to be used to protect administrator accounts
(which count toward the threshold), but will be desirable to average users.

For administrator accounts or other accounts which are expected to contribute
to the threshold, one must devise other techniques.   Fortunately there is
a significant amount of prior work on secure remote authentication and
techniques for using this to hide 
secrets~\cite{deo1998authentication, yang1999password}.   This can be used to 
protect the Shamir Secret Share(s) for the account.
It is possible to view the authentication as decrypting information using a 
secret stored by a remote party (as in fact many such systems are implemented).
We can encrypt the Shamir Secret Share for the user with their key and then
check that the resulting $x$ and $f(x)$ are correct for the user.   

For example, suppose that an administrator has a private key that they use
to login (as is common with SSH private key authentication or smart card
based authentication).   Instead of
XORing the Shamir share with the salted hash of the password, it can
be encrypted with the user's public key.   When the administrator presents
their key to login, this key is used to decrypt the original share.   This
process can be repeated multiple times if the user has multiple shares.
 The share can then be used the same as any other share to 
unlock a PolyPassHash data store.


For authentication of threshold account using other techniques such as 
biometrics, there has been prior work on deriving a private key from this 
(noisy) data~\cite{juels2006fuzzy}.   Once a key is derived from the biometric 
data, the scheme will function identically to the private key authentication 
scheme discussed above.




\subsubsection{Partial Verification}
\label{sec-partial}

When a PolyHashing store restarts, no values may be authenticated until a
threshold are provided.   If the time delay is not important and the rate
at which values are provided is rapid, this may not pose a problem.   However,
in many scenarios, this delay may prove burdensome.

This issue can be addressed with a technique called \emph{partial 
verification}.   Partial verification has the password database
leak partial information about the hash to allow verification before a threshold
is reached.   The core idea is similar for threshold accounts and 
thresholdless accounts, but for simplicity will be explained with threshold
accounts first.   If the Shamir Secret Share is chosen so that it is shorter
than the hash value, this will leak some bytes of the salted hash.
The `partial' hash that is leaked can be used for verification purposes.


\begin{figure}[t]
\center
% We need pdflatex with png images. Plain latex would fail here.
% So, we conditionally include this so we can still build with latex,
% just without this image.
%\begin{minipage}[b]{\linewidth}
\includegraphics[width=1.05\columnwidth]{figures/partialverification.png}
%\end{minipage}
\caption{This figure shows validation using partial verification.
A portion of suffix of the salted hash is stored on disk.  This
allows verification of accounts before a threshold of correct passwords is
provided.
}
\label{fig-partialverification}
\end{figure}

For example, suppose there is a 32 byte salted hash and a 30 byte 
secret (Figure~\ref{fig-partialverification}).   The secret will XOR with
the first 30 bytes, effectively obscuring them.   However, the remaining
2 bytes will consist of the suffix of the salted hash.   When the
system restarts, accounts can be validated using the last 2 bytes of the
hash.   Once sufficient accounts have logged in, the full
hash can be recovered and account authorizations can be 
performed using the full hash as was described in the 
previous section.

Thresholdless accounts can also be validated before a threshold is reached
if partial validation is used.   The technique is similar to threshold 
accounts, where a portion of the suffix of the hash replaces
some of the encrypted hash on disk.   Validation can be 
performed with the suffix until a threshold of correct passwords are provided
and the Shamir Secret Store (and thus the AES key) are recovered.

Increasing the length of the hash which is leaked has a negative 
effect on the resulting data store security.
First of all, suppose that $l$ bits are leaked with via the hash.
This allows an attacker that compromises the password hash database to 
discard $2^l$ possibilities from the password search space.   Thus
if there is a threshold of $k$ and each password has $n$ bits of entropy, 
the attacker can make $k$ passes of cost $2^n$ to remove all but $2^{n-l}$
passwords.   Following this, the attacker will then need to simultaneously
guess from the remaining passwords which will cost an additional $2^{k*(n-l)}$
to crack the store.   (Section~\ref{sec-feasibility} discusses the practical
impact this has on password security.)

However, a small hash also has a negative ramification that
incorrect passwords could log in before a threshold is reached.   In fact,
the probability of a random password working is $\frac{1}{2}^l$.   Rate
limiting the rate of password attempts can mitigate the risk from an attacker
that does not know the salt.   However, if an attacker has stolen the password 
database, they can quickly compute a value that will be correctly verified 
upon restart.  However, once a threshold of passwords is provided,
PolyPassHash knows the full hash and can detect any such
malicious logins.   Much like other decoy schemes~\cite{juels2013honeywords,
Kontaxis_CCS_2013}, partial verification allow incorrect logins but
later detects this occurrence and provides strong evidence that the password
database was leaked.

\eat{
The length of the leaked information need not be the same for every 
account.   There is no reason why the threshold and thresholdless accounts
need to have the same length of leaked information.   In fact, every 
thresholdless account could have a unique quantity of leaked information,
perhaps to take into account entropy in the password.   For threshold
accounts, a variable amount of leaked information could be appended to each
share.   Since the shares hide the secret, leaking hash
information effectively leaks some parts of the shares to the attacker.
We will study the security implications of leaking variable portions of 
threshold accounts in future work.

}







\section{Discussion}
\label{sec-expecteduse}

This section discusses strengths, limitations, and expected uses of 
PolyPassHash as compared to other techniques.


{\bf How does PolyPassHash compare to having an administrator enter a key
at boot time to unencrypt the password database in memory?}

With the administrator entering a key, the system cannot validate passwords 
upon startup until an administrator intervenes.  With partial verification, 
PolyPassHash can process user authentications immediately.   Furthermore, 
if the administrators enter a key, all system administrators would likely 
share this key to be able to quickly act to restart the system.   From 
a security design standpoint, it is considered bad practice to share
passwords.  With 
PolyPassHash different administrators do not need to share a password or key.
Each administrator may have their own password allowing access control to 
happen naturally as administrators join and leave the organization.


{\bf Why not store a key to decrypt the password database in hardware instead?}

Storing a key in trusted hardware (such as a USB 
dongle~\cite{passwordhardwaredongle}) has security benefits,
but also has significant deployment challenges.  
%The software must be rewritten to interface with that 
%manufacturer's secure token.   
If the hardware stops working, the protected data is 
likely to be irrevocably lost.  %Furthermore, 
To have backup servers, it is essential to have identical 
duplicate copies of the secure token.   This complicates use in scenarios 
like cloud computing or even just a standard master / slave deployment.
On the other hand, PolyPassHash is entirely software based and a backup system
can be brought online in the same way that a normal server boots.

{\bf How does an attacker that can read arbitrary memory on the 
authentication server impact PolyPassHash?}

If an attacker can read arbitrary memory on a running server where the 
threshold of authentications has been met, the attacker can recover
the Shamir Secret Store.   This allows the attacker to recover the 
hashes for each password.   An attacker could then crack passwords
individually in the standard way.    While these are not the majority of
disclosed database compromises~\cite{passwordresearchblog,miranteTR13}, 
neither PolyPassHash nor techniques like hardware key storage 
are significantly helpful in this situation.
%protect against situations where the attacker has administrator access on a 
%running authentication system.

However, if an attacker can read
arbitrary memory, the attacker can gain access to plaintext passwords
as they are provided by clients.    This is because typically the client
provides the server with the password, which the
server then combines with the salt to get the hash.  
So without changes to the client software, any server-side
password storage technique will leak plain-text passwords to an attacker
that can read arbitrary memory on a running server.



{\bf How does PolyPassHash change how users interact with existing password authentication systems?}

Using PolyPassHash results in very minor changes to existing password
authentication systems.   
%The only user perceptible difference is when 
%the system is restarted, authentication attempts will be delayed or 
%declined until the threshold is reached.   
%
The client tools for password authentication do not change in any way.   In 
fact, because PolyPassHash only impacts password storage, it is invisible to 
clients.   %A user cannot tell if PolyPassHash is being used, much like it is 
%not possible for a user to tell if a server stores a password using a salted 
%secure hash or in plaintext.
For the administrator's toolchain, the only change is that 
when an account is created, the administrator can specify if the account counts
toward the threshold or not.
%Note that in many cases, such as cloud services, the 
%separation for threshold and thresholdless accounts is clear.   For example, a 
%web service may choose to have a command line interface used by administrators
%to create accounts with a threshold, while the web interface will only
%create thresholdless accounts.

{\bf How does using PolyPassHash integrate with existing password storage 
formats?}

The file format for password storage is similar to existing systems.
For systems which use a database, one can simply insert the share number
as a new column in the table and change the authentication logic to include
PolyPassHash.   

However, 
many servers (such as Linux, Mac and BSD) use the {\tt /etc/shadow} or 
{\tt /etc/master.password} file 
colon delineated formats.
%by `:') such as the username, password expiry date, last date the password was 
%changed, and a single `password' field.
The `password' field contains both the salt and hash but these
different portions of the field are not 
delineated.  
%Instead the storage and verification code knows which 
%bytes of the `password' field are used for each task.   
For PolyPassHash, one can also add the share number at a known byte location
within the `password' field (such as the beginning or end), making this
variable length, opaque string one byte longer.
%We do not anticipate issues with this since the `password' field and length
%are opaque.  (In fact administrators commonly change the length and content of
%the `password' field manually.)   Other fields are separated from the password
%field using `:' as a delimeter and so will not be impacted.


{\bf If there is a single administrator, is PolyPassHash useful?}

There is value in PolyPassHash whenever there are multiple
accounts, even if only one is an administrator.   If an attacker steals the
password file, but does not know the password for the administrator account, 
the attacker cannot crack other user accounts.   
Assuming that the attacker does not know the administrator password
from other sources, such as its use on multiple sites, the 
attacker cannot crack a thresholdless password without cracking 
the administrator password.


{\bf What threshold values are likely to be used?}

%It is logical to assume that most policies for login will request that
Ignoring partial verification,
$k$ of $n$ administrators (or power users) must login to unlock the
PolyPassHash store.   The value of $n$ is largely irrelevant
except it must be smaller than the field size since there are only $n-1$ 
Shamir Secret shares.   (Note $n$ only includes accounts that should count
against the threshold.)
%
%The primary value that matters is $k$, the threshold setting.   
%The threshold value $k$ is the number of individuals 
%that must enter their password before unlocking a data store.   
%A very large value may impact the availability of the system since logins 
%cannot be processed until the threshold is reached.   
We believe that most organizations will have a small 
value for $k$, such as $1$ to $5$.   As our results later show, 
when coupled with non-trivial passwords, even a small value of $k$ 
increases password strength immensely.



{\bf When is PolyPassHash most (and least) valuable?}

When only a single user account exists, such as on a smartphone, 
there is no benefit to using PolyPassHash on the local device.   However, 
essentially all devices are networked and interact with servers.   Web and 
cloud services benefit from PolyPassHash as do any devices that support
multiple user logins.   
Services that manage logins from multiple parties, substantially benefit from
PolyPassHash.     This is because PolyPassHash makes 
a password file compromise only impactful if a threshold of administrator
passwords are already known.   %We believe that any organization that has 
%passwords from multiple users and wants to protect those
%passwords (which should be a universal goal) should deploy PolyPassHash.


{\bf What happens if so many threshold users forget their passwords that a
threshold cannot be reached?}

If there are not enough known threshold users, all password data in the
password file is useless.   However, this does not mean that the device is
unusable.   If used, partial verification will still allow users to log in.   
However, it will not be possible to create new accounts since Shamir
Secret Shares cannot be generated and the AES key protecting the thresholdless
accounts is not known.  Furthermore, mechanisms like root password recovery 
through
the console will still work, allowing any data on the system to be accessed.




%{\bf Why might PolyPassHash be adopted when existing schemes are infrequently 
%used to protect passwords?}
%
%A huge advantage of PolyPassHash is that it is a software-only solution that
%requires no additional hardware or setup.  The only change is in the 
%authentication software.    Furthermore the change is not burdensome (or even
%visibile) to end users.   We believe the power of simply allowing an 
%administrator to improve security through a simple software action (like
%{\tt apt-get install polypasshash}) could substantially increase adoption.



{\bf What happens if users choose extremely weak passwords like 123456, 
letmein, and password?}

Ignoring partial verification for the moment, 
if extremely weak passwords are used for a threshold of threshold accounts 
(like administrators), PolyPassHash will not provide strong protection.   If
there are only a few bits of entropy in the password, while exponentially
larger, the search space will still be small.   It is thus critical that
the administrators choose strong passwords.   However,
if a thresholdless account uses a weak password, the attacker must first
crack a threshold of accounts, thus providing strong protection.

If partial verification is used, then some bytes of the password's 
hash are leaked.   If an extremely weak password is used, this
will be effectively leaked to the attacker.   One should think of partial
verification as reducing the password strength via the number of leaked
bits.   Thus a password with fewer bits of entropy than the partial 
verification length, can be cracked as though it is protected using today's
best practices.


Independent of all of this, extremely weak passwords are susceptable to 
guessing by an external party (without database access) and as best practices
sites should block them from use~\cite{bancommonpasswords}.






\section{Implementation and Limitations}
\label{SEC:implementation}

Our reference implementation for PolyPasswordHasher is available with an MIT
license at \showurlx.  It utilizes a 16 byte salt, with SHA256, to compute password
hashes. The Shamir Secret Sharing routines utilize GF256 as the underlying
field (by encoding each byte as a separate share).  The code base for
PolyPasswordHasher is 120 lines of Python code.  This code handles the
functionality described in Section~\ref{SEC:design}, including support for protector,
bootstrapping, and \thresholdlessaccounts; reading/writing a database to disk;
changing passwords; and detecting \partialverification of incorrect passwords.
PolyPasswordHasher relies on mathematical libraries such as GF256 operations,
Lagrange interpolation, and polynomial math code -- 391 lines of Python and 83 lines
of C code.  Additionally, the implementation uses standard python
libraries for functionalities such as SHA256 and AES.

Several outside developers, unbeknownst to us (until they shared their work
with us) implemented versions of PolyPasswordHasher in Ruby and PHP.  They
created their implementations based on our Python reference implementation and
specification that is available in our GitHub repository.
Table~\ref{TABLE:source-code} describes these outside PPH implementations.
Other external developers have uploaded fixes and improvements to our code when
they wanted it to be compatible with their environment. 

\begin{table}[t]
    \centering
    \renewcommand{\arraystretch}{1.3}

    \begin{tabular}{| c | c | c | }
    \hline
    {\bf Language } & {\bf Lines of Code} & {\bf Author}\\
    \hline
    C & 598 & Local1\\
    \hline
    Python & 391 (+83) & Local2\\
    \hline
    Django & 786 & Local1 + External1\\
    \hline
    Ruby & 437 & External2\\
    \hline
    PHP* & N/A  & External3\\
    \hline
    \end{tabular}
    \caption{Different implementations of \PPH for different Languages and Frameworks, the
    PHP implementation has not been publicly released}
    \label{TABLE:source-code}
\end{table}



\subsection{Handling Alternate Authentication Mechanisms}
\label{SUBSEC:alternate-authentication-mechanisms}

Passwords are not the only mechanism for logging into modern systems. For
example, it is common for users to have a private key to log in over ssh,
biometric-based authentication holds substantial promise, smart cards are often
used to hide user credentials, and single sign-in systems like OAuth and OpenID
are commonplace and provided by many major websites. Any practical protection
mechanism needs to operate in conjunction with such techniques.

Handling non-password logins for shielded user accounts is trivial because
it requires no changes to the system. These non-password logins can be handled
using existing login mechanisms, without any modification. 

For \thresholdaccounts, it is possible to view the authentication process as
decrypting information using a secret that has been stored by a remote party
(as in fact many such systems are implemented). Essentially, we can encrypt the
Shamir Secret Share with the user’s key and check that the resulting share is
correct.  For example, suppose that an administrator has a public/private
keypair that is used to log in (as is common with SSH private key or smart card
based authentications).  Instead of XORing the Shamir Share with the salted
hash of the password, the share can be encrypted with the user’s private key.
When the administrator presents her public key to login, it decrypts the
original share. 

Prior work on deriving a private key from (noisy) data [34] is relevant when
authenticating \thresholdaccounts using other techniques, such as biometrics.
For example, once a key has been derived from biometric data,
PolyPasswordHasher will function in a way that is identical to the private key
authentication scheme discussed in the previous paragraph. 

\subsection{Deployment}
\label{SUBSEC:deployment}

One key issue with deploying PolyPasswordHasher is deciding which accounts should be protector
accounts and which should be shielded.  General guidelines include that
\thresholdaccounts should be assigned to users who choose strong passwords.
Ideally these are users who will log in soon after a reboot, which will
minimize the amount of time needed to bootstrap. Although different servers may
have different use patterns, in our experience, system administrators usually
meet the above requirements.  We examined logs from servers at our institution
and found that system administrators typically login a few minutes after a
system is restarted (likely because they will check the system to make sure it
is operating correctly).  Based on these practices, system administrators tend
to be ideal candidates for \thresholdaccounts.

The threshold setting is another important value for deployment.  In our
experience, a threshold value of 2 - 5 is sufficient, even for systems that
process thousands of password requests.  As we demonstrate in the evaluation
section, even a threshold of 2 increases password strength immensely.

Using PolyPasswordHasher results in very minor changes to existing password
authentication systems. The client tools for password authentication do not
change in any way. In fact, because PolyPasswordHasher only impacts password
storage, it is invisible to clients. For administrators, the only change is
that when an account is created, the administrator can specify whether or not
an account counts toward the threshold.

The file format for password storage is similar to existing systems. In systems
that use a database, the share number and isolated-check bits can simply be
inserted as new columns in the table.  Many servers (such as Linux, Mac, and
BSD) use the /etc/shadow or /etc/master.password colon delimited formats.  As
it is used today, the ‘password’ field contains both the salt and hash but
these different portions of the field are not delineated.  With
PolyPasswordHasher, additional data (such as the share number and
isolated-check bits) can also be added at a known position within the
‘password’ field (e.g., the beginning or end); doing so will lengthen this
variable length, opaque field.




\section{Evaluation}
\label{sec-evaluation}

\begin{figure}[t]
%\begin{minipage}[b]{.45\linewidth}
    \includegraphics[width=.75\columnwidth, angle=270]{resultdata/onebigtimegraph.eps}
	\caption{Time for PolyPassHash operations.  The plots for thresholdless
creation and verification overlap as do the plots for the threshold versions
of each action.}
	\label{fig:time_basic_operations}  
\end{figure}
%\end{minipage}
%\hspace{.06\linewidth}
%\begin{minipage}[b]{.45\linewidth}
%    \includegraphics[width=.75\textwidth, angle=270]{resultdata/onebigperformancegraph.eps}
%	\caption{Performance (operations per second) for PolyPassHash operations.  }
%	\label{fig:performance_basic_operations}  
%\end{minipage}
%\end{figure*}
This section describes a performance evaluation of our prototype of 
PolyPassHash.   
To understand the performance of PolyPassHash,
we performed a series of microbenchmarks on an early-2011 MacBook Pro with 
4GB of RAM, a 2.3 GHz Intel Core i5 processor.  


%\begin{figure}[t]
%    \includegraphics[width=0.3\textwidth,angle=270]{resultdata/onebigtimegraph}
%	\caption{Time for basic PolyPassHash operations.  }
%	\label{fig:time_basic_operations}  
%\end{figure}
%
%\begin{figure}[t]
%    \includegraphics[width=0.3\textwidth,angle=270]{resultdata/onebigperformancegraph}
%	\caption{Performance (users impacted by an operation per second) for basic PolyPassHash operations.  }
%	\label{fig:performance_basic_operations}  
%\end{figure}

All operations are the mean verification time across 100 runs and
are performed with the password file already present in
memory.  
For benchmarking purposes, each action is performed sequentially despite being 
embarrassingly parallelizable.


\subsection{Performance Of A PolyPassHash Store}

%To evaluate the performance implications of PolyPassHash, we varied the 
%threshold.   While we do not believe that
%there will be many (if any) practical situations where the threshold would be 
%set to a value larger than ten.
%However, for completeness, we show the result of varying the threshold
%across a broader range of possible values.   
Figure~\ref{fig:time_basic_operations} shows the time taken by
different operations (discussed below).  Unless noted below,
the time of the operations does not depend on other factors such as the 
number of accounts in the password database.




{\bf Account Verification.}
The mean verification time of a threshold account
varies from 57$\mu$s to 163$\mu$s depending on the
threshold.  With a threshold of 8, this allows the verification of 
more than 16K user accounts per second.

Thresholdless accounts are verified in a time that is independent
of the threshold size because it merely computes    
a salted hash and performs an AES encryption operation.
Verifying
a thresholdless account takes about 29$\mu$s, allowing
about 35K such actions to be performed each second.   
In comparison, the current
best practice of generating the SHA256 hash of a known salt is a little over
3$\mu$s on the same hardware, which allows hundreds of thousands of
account authentications per second.   
%Note if there were a case where the account 
%validation software needed to process more than 35K account authorization
%attempts per second, the process is embarrassingly parallelizable.



{\bf Account Creation / Password Change.}
The account creation time is similar to that of password verification.
Depending on the threshold, the average verification time 
varies from 77$\mu$s to 190$\mu$s.  
A store with a threshold of 8 can create more than 12K 
accounts per second.   Given there are a maximum of 255 threshold accounts
that can be created (at least with a PolyPassHash implementation that uses
GF256), this clearly is not a performance concern.


Much like account verification, thresholdless accounts can be created in 
time that is independent of the threshold.   
A thresholdless account creation takes about 25$\mu$s.   As a result, 
about 40K thresholdless accounts can be created each second.   This is
similar to the time it takes to generate a salted SHA256 hash for a 
provided password (16$\mu$s). 

Note that changing the password requires that PolyPassHash performs the same
operation as account creation (potentially along 
with authentication of the old password).


{\bf Initializing a Store.}
The time to create a store varies depending on the threshold.   This
cost is dominated by generating cryptographically-suitable random
numbers.   By varying the threshold, the creation time varies from 
380$\mu$s to 1.13ms.   This operation is only performed when a 
new password file is created.

{\bf Unlocking a PolyPassHash Store.}
When the server restarts, the
random coefficients are computed from the set of provided shares with
full interpolation.   The time needed varies 
between 202$\mu$s and 2.87s as the threshold changes since
the number of polynomials changes as the store size grows.
Note that small thresholds are very fast, with a store with a threshold of 8
being unlocked in 617$\mu$s.

This is fast enough that if some passwords are incorrectly entered, 
PolyPassHash can (na\"ively) detect them.   For example, suppose that
a threshold of 10 is used and that 14 passwords are provided and only
10 of them are correct.   All possible combinations of 10 passwords can be
checked by PolyPassHash in 618ms.
Partial verification will also discard most passwords that are entered 
incorrectly to prevent them from being used to attempt to unlock the 
password store.
% If there is a huge threshold with a large number of
%candidate logins and passwords, it would also
%be possible to apply more advanced techniques for detecting incorrect
%shares~\cite{carpentieri1995perfect,Yang_compsac_02,devet2012optimally}.


%Furthermore, this process needs only to be performed when the 
%system is restarted.  


\subsection{Memory and Storage Costs of PolyPassHash}


\begin{table}[t]
{\scriptsize
\begin{tabular}{|l|l|l|l|}
\hline
Password & Original & Salted, Hashed & PolyPassHash \\
Source & Disk Space & Disk Space & Disk Space\\
\hline
\hline
eHarmony* & 51.6MB & 100MB & 102MB \\
\hline
Formspring* & 27.3MB & 34.8MB & 35.2MB \\
\hline
Gawker & 75.2MB & 119MB & 120MB \\
\hline
LinkedIn* & 252MB & 424MB & 430MB \\
\hline
Sony & 2.98MB & 4.95MB & 5.00MB  \\
\hline
Yahoo & 17.8MB & 35.0MB & 35.4MB \\
\hline
\end{tabular}
}
\caption{Disk space to store password data from different account disclosures.
Entries with a * indicate only the password hash portion of the database
was leaked.   Other breaches include usernames and similar data.  }
	\label{tab:extrahashcost}  
\end{table}



%\begin{table}[t]
%{\scriptsize
%\begin{tabular}{|l|l|l|}
%\hline
%Storage Mechanism & Disk per entry & Total memory cost\\
%\hline
%\hline
%Salted Secure Hash & 48 bytes & N/A (stored on disk) \\
%\hline
%Threshold & 49 bytes & $k*32$ bytes (likely $<1$KB) \\
%\hline
%Thresholdless (separate file) & 48 bytes & N/A (included in threshold) \\
%\hline
%Thresholdless (same file) & 49 bytes & N/A (included in threshold) \\
%\hline
%\end{tabular}
%}
%\caption{Cost to store the password hash database for PolyPassHash and 
%conventional salted hash techniques in terms of memory and disk space.
%A 16 byte salt and 32 byte hash is assumed for all schemes.}
%	\label{tab:extracost}  
%\end{table}

Storing passwords with PolyPassHash requires additional information be stored
for each account.   Namely, for each account, since each share is a value
in GF256, there must be a one byte share number stored for each account.   
This represents an additional 1 byte of storage space for each account in 
addition to the cost of current hash techniques.   This
has a minimal impact on the disk space needed to store production
password databases (Table~\ref{tab:extrahashcost}).   If the thresholdless 
accounts are stored in a separate file or are
otherwise distinguished, then only threshold accounts require the extra
byte of storage.   This will result in a cost for those accounts that is
identical to the salted, hashed scheme.

In addition, any account that has 
multiple entries (to count multiple times toward the threshold), requires an 
additional salt (16 bytes), hash (32 bytes), and share number (1 byte)
for every entry after the first.
Since there are at most 255 Shamir secret shares in GF256, this can be treated 
as a fixed cost of a few kilobytes.
%We do not believe the additional storage 
%cost will prove a burden in any practical scenario.

There is additional memory cost because the server must cache
the polynomial coefficients for a store.   The total
size of this data is the threshold value multiplied by the hash length.
Thus, the in-memory threshold data is likely to be under a 
kilobyte in practice.% (Table~\ref{tab:extracost}).   This is likely to be smaller than the (also minimal)
%memory space needed to store the additional code for PolyPassHash.
When thresholdless accounts are used, the AES key used is the 
constant term 
of the polynomial coefficients and so does not incur additional cost.
The use of partial verification has essentially no impact on the storage or 
memory requirements.

%Note that if thresholdless accounts are used and stored in the same
%file as threshold accounts, the disk costs
%are the same for both types of accounts (Table~\ref{tab:extracost}).   
%In order to indicate an account is thresholdless while using the same
%password database, the share number can be 
%set to 0 (an invalid share in traditional Shamir Secret Sharing).
%%There is the same memory and disk storage costs for each account regardless
%%of whether or not it counts toward the threshold.
%This indicates the account is thresholdless while permitting storage
%in the same file.




%\subsection{Thresholdless Accounts}
%
%\cappos{AES can compute about 400K 256 byte decryptions per second on my 
%laptop.   This is much faster than even SHA256, so should be much faster than
%accounts with a threshold.}


\subsection{Feasibility Of Cracking Passwords}
\label{sec-feasibility}

\begin{figure}[t]
\hspace{-4mm}
\includegraphics[width=.36\textwidth, angle=270]{resultdata/plotcrack.eps}
	\caption{Time to crack a password store given 1 billion attempts
per second.   The k=1 case for PolyPassHash is the same as the
legacy salted hash scheme.   The partial verification values show 
how leaking 2 bytes impacts cracking time.   A threshold of 3 without partial 
verification falls well outside the axis of the graph.}
	\label{fig:cracktime}  
\end{figure}

%Assuming a threshold of accounts is not known, cracking passwords takes 
%exponentially more time with PolyPassHash than standard hash
%schemes.   The reason is that an attacker must simultaneously attempt to
%crack the threshold number of passwords at once.
%An attacker must attempt to determine the random coefficients that are 
%protected by Shamir Secret Sharing before validating passwords.   This means 
%the attacker must know enough passwords to have at least the 
%threshold number of passwords.
%If an attacker needs to guess $p$ passwords, where each password has $V$ 
%possible values, the attacker is choosing values from $V^p$ possible guesses.   
%Solutions that involve slower hash functions or multiple hash iterations 
%result in only a linear increase in attack time and verification time.
%Since passwords must be correctly guessed simultaneously with PolyPassHash, 
%PolyPassHash results in an \emph{exponential} increase in attack time.

{\bf General Analysis.}
Even when an attacker needs to guess only a few passwords, in many cases 
PolyPassHash's \emph{exponential increase} in guessing time ($O(V^p)$ instead 
of $O(pV)$) makes guessing computationally infeasible 
(Figure~\ref{fig:cracktime}).   For example, suppose 
an attacker wants to guess three passwords that they know are each comprised 
of 6 randomly chosen characters\footnote{These passwords
are extremely weak and we do not advocate their use.   This 
example is used to illustrate the strength of PolyPassHash even with 
weak passwords.}.  Recent results 
have shown that a GPU can compute on the order of a billion password hashes a 
second~\cite{ElcomSoftGPUCracking, zonenberg2009distributed}, allowing the 
attacker to search the key space for all three passwords in under an hour.
Thus the current state of the art provides little protection in this case.

In the case of PolyPassHash, the attacker must \emph{simultaneously}
guess all three passwords.   This means the attacker needs to guess from 
$3.97 \times {10^{35}}$ values.   This is roughly 23 orders of magnitude more effort.
When PolyPassHash is used with the same passwords and the GPU 
accelerated technique, searching the key space would take $1.25 \times 10^{19}$
CPU years (does not fit on the axis of Figure~\ref{fig:cracktime}).   To put 
this number into context,
ignoring smartphones there are about 900 million computers on the 
planet~\cite{computersexisting}.   The estimated age of the universe is 
$(1.3798 \pm 0.0037) \times 10^{10}$~\cite{universeage}.   The time needed to crack 
three random 6 character passwords protected
by PolyPassHash is more CPU time than would be provided by every computer 
on the planet working nonstop for the estimated age of the universe!

Even a threshold of two is substantially stronger than existing
best practices.   Searching the key space would require over 17 million CPU 
years of effort.

%If partial verification is used, this effectively allows accounts to be
%compromised by reducing the password entropy by the number of bits
%leaked.   However, assuming the remaining number of bits is positive, the
%passwords must then be simultaneously cracked.   In essence the attacker
%precomputes a candidate list of likely allowed passwords based upon the leaked
%data and can then try all combinations of those passwords.

{\bf Partial verification.}
Partial verification allows an attacker to first reduce the search space for
a specific account by eliminating accounts that do not match the leaked bits.
If the attacker knows the password pattern, the attacker can first 
precompute all passwords that match that pattern and hash.   This requires
the same amount of effort ($k*2^n$) as cracking passwords when stored with 
traditional best practices.   If the attacker has sufficient space to
store these passwords, the attacker then may use combinations of
these precomputed passwords to try to unlock the store.  This allows
the attacker to search the space for the Shamir Secret Store in $2^{k*n-l}$ 
guesses. 
Each byte used for partial verification effectively
reduces the strength of a random password by approximately 1.22 characters.
However, the time needed to unlock the Shamir Secret Share still represents
an exponential increase and dominates the overall cost.

For example, if 2 bytes are leaked for 6 random character
passwords (Figure~\ref{fig:cracktime}), the attacker will need to do
735 trillion operations for each of the
three passwords and then $1.41 \times 10^{21}$ operations to crack the password
store.   With one billion operations per second, sweeping the search
space would still require 45 thousand CPU years (8 orders of magnitude
more time than salting and hashing).
%password strength if used, even with partial verification PolyPassHash still
%represents a huge benefit over only salting and hashing.






{\bf Case Study.}
To explore how these results hold for real passwords, we performed an
experiment using the password data dumped from the Sony account breaches
\cite{sonyhack}.   Of the leaked passwords known to the authors
(Table~\ref{tab:extrahashcost}), this is the only data set which 
explicitly lists which accounts are administrator accounts versus normal users.

In the database dump, there is password data for both outside
user accounts as well as accounts have administrator access.   (There is
also a database with testing accounts which we ignore.)
The four administrator accounts have passwords with the estimated
entropy in bits~\cite{passwordstrength}:
password@1 (5.322 bits), %Annelies (17.653 bits), 
welkom@1 (26.553 bits),
waderobsen (30.618 bits), %foto4U2 (31.186 bits), fietspomp@1 (42.139 bits),
and itsafullcyrcle (44.011 bits).   
Given a rate of checking a billion password hashes a second, the first 
three passwords can all be cracked in two seconds.   The
remaining password (and thus every administrator password) could be
cracked in under 5 hours.
%Using John the Ripper, version 1.7.9-jumbo-7 built for macosx-x86-64, with
%the default settings on a early 2011 era Macbook Pro, allowed the passwords
%to be cracked in .
%In comparison, 948K of the 6.5M unique LinkedIn passwords were cracked within
%1 hour of computation on the same system / program.

If PolyPassHash were protecting these passwords, the ability to crack the
passwords depends on the threshold.   The effective password strength is 
that of the weakest passwords combined, times the probability of selecting
those accounts in that order.  
A threshold of 3 will have an effective entropy of 67.078 bits,
which will take nearly 5000 years to crack at 1 billion guesses a second. 

With a threshold of 2, the effective entropy
is 35.459 bits, which can be cracked in under a minute.   
However, if all administrators had chosen a password
as strong as the password waderobsen (which is considered a weak
password by best practices~\cite{passwordlength}), then the password would 
have had an effective 
entropy of 64.820 bits, which will take more than a thousand CPU years to crack
at 1 billion guesses per second.   This underscores that administrators
still should not choose immensely poor passwords like password@1.   
Independent of password storage, avoiding trivially guessable passwords 
is essential to prevent remote brute-force password cracking.

The thresholdless user passwords (many of which are extremely poor) 
cannot be cracked until a threshold of administrator passwords are 
compromised.  This means that even if only administrators can be convinced to 
use strong passwords, PolyPassHash provides substantial security benefits.





\eat{
% I like this, but it isn't really true given hashing validation of shares.
When strong passwords are used, the problem becomes even more intractable.   
In fact, with sufficiently strong passwords, it can be impossible to know that
one has correctly guessed the password.   For example, assume the attacker
knows one fewer than the threshold number of passwords and must guess only one 
other to know the random coefficients.   If the password hash and length is 
such that any possible hash value can be generated and all are equally likely 
(intended properties of hashing schemes), then all hashes are equally
likely.  This means in the extreme case where the password length is such that 
all hashes can be generated and all are equally likely, PolyPassHash's use of 
Shamir Secret Sharing provides information-theoretic protection.
}


\section{Related Work} \label{sec-related}

%\cappos{ cite: \url{http://cs.gmu.edu/~csnow/library/unix/Klein_passwd.pdf} 
%and lots of other things}

%\cappos{A nice survey paper with related work details~\cite{tsai2006password}.}
There has been extensive work on password security stretching back many
years~\cite{morris1979password,klein1990foiling, florencio2007large}.   
Password database disclosure is a problem that, if anything, seems to be
more prevalent as time goes on~\cite{clair2006password,miranteTR13,passwordresearchblog}.
Password security has been the focus of much study with many promising 
solutions solving different portions of the problem~\cite{tsai2006password}.

Our work on PolyPassHash is unique in that is the first password database
protection scheme that:
\begin{enumerate}
\item assumes the attacker can read all persistent storage,
\item requires only a software change on the server,
\item and requires exponentially more time for the attacker to crack passwords.
\end{enumerate}

PolyPassHash can be deployed with minimal server changes and without modifying 
clients at all.   

{\bf Multi-server Password Authentication.}
There are a wide variety of authentication schemes that use multiple servers
to store password data~\cite{Chai20071046,bagherzandi2011password, katz2005two}.
The assumption is that the attacker cannot compromise a threshold of
the servers.  In contrast, PolyPassHash uses a single server but uses
a threshold system to hide information that can only be unlocked
with a threshold of correct user passwords.

{\bf Decoys.}
Recently, researchers have suggested multiple techniques that use a set of 
extra password entries~\cite{Kontaxis_CCS_2013, juels2013honeywords}.  For 
example, the Honeywords~\cite{juels2013honeywords}
system uses a separate server holds information about which password entry is 
correct.   If an 
attacker obtains the password database then they do not know which password 
entry is correct.   Entering a password which matches the hash of a 
different password entry will trigger an alarm which notifies the 
administrator of a password hash file breach.   However, for this 
to work, there must be a separate, secure server which authenticates 
the index of that entry in the file (a one byte value).   
The real value of HoneyWords is that it can also operate when that server is
offline and / or check passwords at a later time to detect breaches later.
%If it is 
%possible to build such a system, it is not clear why that server 
%cannot simply securely store and check the password hash (a fixed 
%size value of about 32 bytes).  
PolyPassHash utilizes ideas from these works in constructing the partial
verification technique discussed in Section~\ref{sec-partial}.

{\bf Key Stretching.}
One way to mitigate the effectiveness of password hash cracking is to
use techniques for key stretching~\cite{kelsey1998secure}.   This involves
performing multiple rounds of cryptographic operations to validate a key.
This effectively slows down both the attacker's cracking of passwords and the
user's authentication by the same factor.   In contrast, PolyPassHash results 
in an exponential increase in the amount of time needed by requiring multiple
passwords to be simultaneously guessed.  Key stretching is orthogonal
to PolyPassHash and could be trivially used in conjunction.
%that knows a threshold of correct account authentications.


%{\bf Hardware-based Password Database Encryption.}
%Several companies use password database protections that center
%around storage of keys or passwords in hardware, such as through a USB dongle.
%Hardware based solutions are fundamentally more 
%secure than PolyPassHash since even if an attacker has access to all memory,
%the password database is not at risk.   
%Unfortunately, these systems are not widely deployed.
%One issue they face is that each device presents its own interface for password
%storage and decryption.   A company that wants to deploy a solution must 
%provide new authentication code for the specific device and also purchase
%the devices.   
%If the hardware token stops workting, the data may be irrevocably lost.
%If the server storing a device is damaged or crashes, it 
%is not possible to bring up a backup using the same authentication 
%database.  On the other hand, PolyPassHash is a purely software-based solution.
%This not only makes it easier to deploy, but also means that a backup can be 
%brought online in the same way that the normal server boots.   As a result
%PolyPassHash is much easier to deploy and manage in distributed computing
%environments such as in cloud computing.

%\cappos{Read over and add cites.}

{\bf Bounded Attacker.}
Di Crescenzo~\cite{di2006perfectly} proposed a scheme for protecting
password data when an attacker can only read a bounded amount of data from 
storage.   This works by an organization configuriing
network monitoring hardware and setting up a separate server to process 
authentication requests.
In the widely published password file compromises, the 
attackers were able to steal complete password file 
data~\cite{miranteTR13,passwordresearchblog}.   In contrast, 
PolyPassHash requires no network changes or monitoring and works even when an 
attacker has complete access to stored information, such as a disk backup.

Prior work by Gwoboa~\cite{gwoboa1995password} hides passwords using
a trapdoor function (public key cryptography) and techniques from threshold
cryptography.   It can authenticate users with two hidden pieces of 
information, a user ID (likely not the user name for security reasons) and 
the password.   However, a major concern of the scheme is how the private
key is stored on the server.   The authors propose splitting it amongst 
multiple systems and using threshold cryptography.   Clearly if all 
persisted data is known, this key (and thus all passwords) are at risk.
In PolyPassHash, as long as a threshold of passwords are not known, all 
persisted data can be stolen by an attacker without compromising user 
passwords.


{\bf Biometrics.}
Biometric authentication has substantial promise for secure
authentication~\cite{atallah2005secure, snelick2005large, tuyls2004capacity, 
boyen2005secure, erkin2009privacy, kerschbaum2004private, osadchy2010scifi,
monrose2001cryptographic, sae2012biometric}.
There has been a substantial amount of work on how to store and authenticate 
users with this information.   Like PolyPassHash, some of these systems 
use a threshold of information to validate and authenticate users, in part
to deal with noisy biometric data~\cite{juels2006fuzzy, ballard2008practical}.  
While users must still remember a password to use PolyPassHash, it does not 
require client-side hardware.

Prior work uses keystroke dynamics to change stored password 
data~\cite{monrose2000keystroke}.  This technique relies on reading timing 
information from when the user types their password into a site.   This 
provides promising protections, but requires changes to the client and server 
to correctly operate.   In comparison, PolyPassHash protects password
hash information with no change to the client and minimal server changes.

{\bf Authentication Using Tokens or Smart Cards.}
Much authentication has looked at authentication in the context of
banking~\cite{deo1998authentication, yeh2010two}, health 
services~\cite{ahn2002towards},
or a more general context~\cite{chien2002efficient, yang1999password}.   These 
systems are
extremely effective and are widely used for banking and protecting access
to classified systems.   Unfortunately, these devices incur a per-user 
cost and thus are not often used in contexts where the user and server have
no prior commercial or authentication relationship.   PolyPassHash can 
be applied to webmail systems and social networks where this relationship
does not exist a priori.

{\bf Two-Factor Authentication.}
The use of two-factor authentication~\cite{di2005two} is provided by some 
popular services (typically through a user's smartphone).   The use of 
two-factor authentication does not
change PolyPassHash's use in any way.   Users can easily get the best of both
protections by a simultaneous deployment of each technology.



{\bf Multiparty Computational Authentication.}
There are a variety of schemes that perform secure, remote authentication using
computation by the client and server on legacy hardware~\cite{wu1998secure,
Lomas_SOSP_89, chien2001modified, jan1998paramita, gong1995optimal, 
camenisch2010credential, brainard2003nightingale,
katz2001efficient, katz2003forward, gong1993protecting}.  These schemes have 
significant positive aspects such as (in some cases) requiring an attacker to 
be online to validate communications.   However, they require multiparty 
protocols which require changes on clients and servers.   They also do not 
function in non-distributed scenarios.   PolyPassHash works with no 
changes to clients, minimal visible changes to the administrator and
operates on a single system.


{\bf Related Key Exchange Schemes.}
There are also many systems for secure key exchange~\cite{shoup1999formal} 
such as Pass\-word-Au\-then\-ti\-cated Key Exchange
(PAKE)~\cite{boyko2000provably, shen2010towards,sathik2010secret, 
jablon1996strong}, Encrypted Key Exchange (EKE)~\cite{steiner1995refinement,
lucks1998open, jablon1997extended}, and further 
enhancements~\cite{wang2005strengthening}.   These 
systems allow parties that
share a password to securely find an encryption key to hide communications.   
These systems provide excellent protection and can handle compromises
of the memory of systems in some cases.   However, unlike PolyPassHash, they 
typically involve multi-round authentication and require changes to both the 
client and server.   

% JAC: Too much detail.
%
% Using this scheme with thresholdless accounts is trivial and follows the
% same pattern as 5.2.2.   These schemes can be used with threshold accounts
% by having the server and client choose a random string the length of a
% shamir secret share.   The server stores the share XOR this string.
% After authentication, the client sends this string to the server and the
% server can recover the share.


{\bf Helping Users Choose Stronger Passwords.}
There have been many efforts to help users to choose stronger,
more memorable passwords or expose weak 
passwords~\cite{topkara2007passwords,klein1990foiling,bishop1995improving,
schechter2010popularity,
komanduri2011passwords, shay2010encountering, xkcdpassword}.   These 
techniques can be very effective at protecting users,
but must be adopted by users.
PolyPassHash provides an exponential improvement in protection for user 
passwords.   Assuming threshold passwords are strong
(which these methods assist with), PolyPassHash strengthens the
protection of stored user passwords.   This is especially critical when
partial verification is used.

{\bf Password Managers.}
There are a myriad of password managers that help users choose secure
per-site passwords including LastPass, 1Password, and OnePass.   These
systems store password data and lock it using a user's credentials.   As a
result, it is the case that the third party software (and often their server)
will know the user's password.

To mitigate this, several groups have proposed cryptographic techniques
to take a user's password and generate secure, per site 
passwords~\cite{halderman2009lest,ross2005stronger, halderman2005convenient}.
These techniques are effective (and more secure) but can create passwords that 
are incompatible with the server's password policy.

These techniques require client side changes (only) while PolyPassHash
requires only server side changes.   Both can (and should) be used safely 
in conjunction for improved security.


{\bf Single Sign-On.}
Single sign-on systems like OpenID and OAuth have promise for organizations to
securely offload authentication
to a third party.   This proves convenient for users, but is far from 
ubiquitous for a variety of reasons~\cite{sun2010billion}.  These
systems have some security issues~\cite{openidsecurity,oauthsecurity}, but
overall can be effective when properly used.   PolyPassHash integrates 
cleanly with non-administrator logins for such systems.   In addition,
PolyPassHash can be used by the Single Sign-On provider to provide 
security to the authenticating users.

{\bf Non-password Authentication.}
Many researchers have proposed authentication based upon non-password items
such as pictures~\cite{dhamija2000deja}.   In practice, these
systems can have security limitations if users do not appropriately choose
their authentication tokens~\cite{davis2004user}.   For exotic 
authentication mechanisms like this, PolyPassHash functions well for 
non-administrator accounts, requiring no changes to the system.



% JAC: It looks like ``A New Threshold Password Authentication Scheme'' isn't
% available.   


%\cappos{There are schemes where people do crypto operations~\cite{hopper2001secure}.   Totally crazy!}



%\input{discussion}

\section{Conclusion}
\label{sec-conclusion}

This work addresses an important problem in the security community by 
protecting user passwords in the event of password file disclosure.   
Previously an attacker that obtained a salted and hashed password file could 
easily crack passwords one at a time.   Now,
assuming the attacker must have access to a threshold of passwords to 
crack individual passwords.  As a result,
password cracking requires exponentially more work for the attacker and
is infeasible in many cases.

Our implementation of PolyPassHash demonstrates that the storage and
performance properties are similar to systems that are widely used in practice.
PolyPassHash integrates naturally with alternative authentication 
mechanisms, tools, and techniques.
In on-going work, we are deploying PolyPassHash %in production 
%use 
in several domains including a production web service used by thousands
of users and a cloud computing infrastructure.   
%Our future work will involve studying any
%usability concerns that arise from PolyPassHash's adoption.

Our reference implementation is available with an MIT license at:
\showurlx.
%\url{https://polypasshash.poly.edu}.   

\cappos{Future Work: partially rotate share coverage after unlock to 
minimize attack effectiveness after reboot.

Future Work: Use variable quanta shares to incrementally protect more data.}
%\subsection{Future Work}
%
%
%In our on-going work, we are extending upPIR to better mask file sizes, 
%especially for large updates.   We are exploring different algorithms
%that the vendor may use to pack content into a release.
%We are also extending upPIR
%to recover gracefully from mirrors that serve bogus content.   Our efforts
%also include applying upPIR more broadly than software updates.
%
%We also are pursuing an open standard for upPIR communications to ensure
%interoperation between different client, mirror, and vendor tracker 
%implementations.






% trigger a \newpage just before the given reference
% number - used to balance the columns on the last page
% adjust value as needed - may need to be readjusted if
% the document is modified later
%\IEEEtriggeratref{8}
% The "triggered" command can be changed if desired:
%\IEEEtriggercmd{\enlargethispage{-5in}}

% references section

% can use a bibliography generated by BibTeX as a .bbl file
% BibTeX documentation can be easily obtained at:
% http://www.ctan.org/tex-archive/biblio/bibtex/contrib/doc/
% The IEEEtran BibTeX style support page is at:
% http://www.michaelshell.org/tex/ieeetran/bibtex/
%\bibliographystyle{IEEEtran}
% argument is your BibTeX string definitions and bibliography database(s)
%\bibliography{IEEEabrv,../bib/paper}
%
% <OR> manually copy in the resultant .bbl file
% set second argument of \begin to the number of references
% (used to reserve space for the reference number labels box)
%\begin{thebibliography}{1}
\bibliographystyle{acm}

%\bibitem{IEEEhowto:kopka}
%H.~Kopka and P.~W. Daly, \emph{A Guide to \LaTeX}, 3rd~ed.\hskip 1em plus
%  0.5em minus 0.4em\relax Harlow, England: Addison-Wesley, 1999.
\bibliography{bibdata}

%\end{thebibliography}




% that's all folks
\end{document}


