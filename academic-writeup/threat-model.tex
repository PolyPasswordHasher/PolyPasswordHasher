\section{Threat Model}
\label{SEC:threat-model}

We base our threat model on salted hashing, which is the most widely deployed
protection scheme.  In salted hashing, the user provides a username and
password to the server.  If a user is locally at the server, this is typically
done using a keyboard attached to the device.  If a user is remote, the
password is usually provided over an encrypted channel, such as HTTPS or SSH.
Either way, the username and password are input to the server in plain-text.   

For a server to employ salted hashing, the only requirement is that there is
software on the server that implements both salting and hashing.  Notably,
there are no hardware requirements for the server (e.g., hardware tokens or
additional servers) and no client software is needed.  
    
We assume that:

\begin{itemize}

    \item An attacker can read all data that is persisted on disk, including
    the password database.  The attacker cannot read arbitrary memory on the
    server.  As with any scheme that does not require client changes, if an
    attacker can read arbitrary memory, the attacker can observe the passwords
    in plain-text. (Fortunately, most reported password database disclosures
    did not involve an attacker with access to arbitrary memory
    \cite{miranteTR13, passwordresearchblog}.)

    \item The server will restart periodically. All data kept in memory is lost
    at this point and the server must restart using only the data on disk -- which
    is attacker visible. 

    \item The attacker may have a priori knowledge of correct passwords for
    some user accounts -- for example, on sites that allow outside users to register
    accounts.

    \item We assume that well known and widely used cryptographic primitives, such as
    SHA256 and AES, are not breakable by the attacker.

\end{itemize}

Alternative solutions that use this threat model as well as alternative threat models are 
discussed in Related Work (Section~\ref{SEC:related-work}).

